\documentclass[11pt]{article}
\usepackage{mon_paquet}

\usepackage{marvosym}
\usepackage{mon_paquet}
\usepackage{amssymb}

\usepackage{hyperref}
\usepackage{comment}

\RequirePackage[left=1cm,right=1cm,top=1cm,bottom=1cm,noheadfoot]{geometry}
\RequirePackage{fancyhdr}\pagestyle{fancy}
\lhead{Formation Python}\rhead{Année Scolaire 2018-2019}\lfoot{}\rfoot{\LaTeXe{}}
\renewcommand\headrulewidth{0pt}%ligne en dessous de l'en tête


%graphes, physique-chimie
%\usepackage{pst-osci}\usepackage{pst-diffraction}\usepackage{pst-circ}\usepackage{pst-dosage}\usepackage{pst-labo}\usepackage{pst-optic}\usepackage{pst-spectra,pstricks-add}\usepackage[pictex]{m-ch-en}\usepackage{m-pictex,m-ch-en}
%\usepackage{psfrag}\usepackage{color,colortbl}\usepackage[table]{xcolor}\usepackage{graphicx}\usepackage[usenames,dvipsnames]{pstricks}\usepackage{epsfig}\usepackage{pst-grad}\usepackage{pst-plot}

%mise en page, multicolonnage, paysage, tableaux maths
%\usepackage{array}\usepackage{multirow}\usepackage{lscape}\usepackage{multido}

%paquets divers, maths, encadrement, symboles, cursif, qcm,lettrine,ombrage
%\usepackage{lettrine}\usepackage{eurosym}\usepackage{amsmath,amssymb,mathrsfs}\usepackage{esvect}\usepackage{esdiff}\usepackage{cancel}\usepackage{fancybox}\usepackage{shadow}\usepackage{pifont}\usepackage{fourier-orns}\usepackage{frcursive}\usepackage{bm}\usepackage{alterqcm}\usepackage{pstricks,pst-3d}(pour les ombres)\usepackage{enumitem}
%\renewcommand\thesection{\Roman{section}}
\usepackage{hyperref}\usepackage{esvect}\usepackage{bm}
\title{Formation Juin 2019 :\\$\star \star \star$\\Python pour l'enseignant de Physique-Chimie au lycée\\$\star \star \star$\\Bonus 1}\author{}\date{}
\begin{document}
\maketitle
\thispagestyle{fancy}






% Commandes de base
\newcommand{\from}[0]{\textcolor{orange}{from }}
\newcommand{\import}[0]{\textcolor{orange}{import }}
\newcommand{\for}[0]{\textcolor{orange}{for }}
\newcommand{\while}[0]{\textcolor{orange}{while }}
\newcommand{\dans}[0]{\textcolor{orange}{in }}
\newcommand{\si}[0]{\textcolor{orange}{if }}
\newcommand{\sinon}[0]{\textcolor{orange}{else }}
\newcommand{\True}[0]{\textcolor{orange}{True }}
\newcommand{\False}[0]{\textcolor{orange}{False }}
\renewcommand{\si}[0]{\textcolor{orange}{if }}
\renewcommand{\sinon}[0]{\textcolor{orange}{else }}
\newcommand{\et}[0]{\textcolor{orange}{and }}
\newcommand{\ou}[0]{\textcolor{orange}{or }}
\newcommand{\as}[0]{\textcolor{orange}{as }}
\newcommand{\defi}[0]{\textcolor{orange}{def }}
\newcommand{\return}[0]{\textcolor{orange}{return }}


\newcommand{\range}[0]{\textcolor{violet}{range}}
\newcommand{\chev}[0]{\textcolor{violet}{\scriptsize $\bm{>>>}$ }}
\newcommand{\print}[0]{\textcolor{violet}{print}}
\newcommand{\type}[0]{\textcolor{violet}{type}}
\newcommand{\dir}[0]{\textcolor{violet}{dir}}
\newcommand{\len}[0]{\textcolor{violet}{len}}
\renewcommand{\input}[0]{\textcolor{violet}{input}}
\newcommand{\float}[0]{\textcolor{violet}{float}}
%\renewcommand{\int}[0]{\textcolor{violet}{int}}
\newcommand{\abs}[0]{\textcolor{violet}{abs}}

\newcommand{\tab}[0]{\qquad \quad\,\, }
\newcommand{\tabis}[0]{\quad\, }


\newcommand{\str}[1]{\textcolor{green}{#1}}

\renewcommand{\com}[1]{\textcolor{red}{#1}}

%Environnement Python : 
 \newenvironment{python}[1]{\begin{center}\begin{normalsize}\begin{sffamily}
 \begin{minipage}{#1\textwidth}%
\hrulefill~\raisebox{-0.8ex}{{\Large \textcolor{red}{\Keyboard}} \quad\textcolor{red}{\textit{\textsf{code Python}}}\quad {\Large\textcolor{red}{\Keyboard}}}~\hrulefill\par}%
{\par\hrulefill \end{minipage}
\end{sffamily}\end{normalsize}\end{center}}




%Environnement correction Python : 
 \newenvironment{correc}[1]{\begin{center}\begin{normalsize}\begin{sffamily}
 \begin{minipage}{#1\textwidth}%
\hrulefill~\raisebox{-0.8ex}{{\Large \textcolor{red}{\Keyboard}} \quad\textcolor{red}{\textit{\textsf{\textsc{une correction possible}}}}\quad {\Large\textcolor{red}{\Keyboard}}}~\hrulefill\par}%
{\par\hrulefill \end{minipage}
\end{sffamily}\end{normalsize}\end{center}}

















% Autres : quasi pas utilisé depuis l'environnement Python
\newcommand{\ora}[1]{\textcolor{orange}{\texttt{#1}}}
\newcommand{\vio}[1]{\textcolor{violet}{\texttt{#1}}}
\newcommand{\rou}[1]{\textcolor{red}{\texttt{#1}}}




%%%%%%%%%%%%%%%%%%%%%%%%%%%%%%%%%%%%%%%%%%%%%%%%%%%%%%%%%%%%%%%%%%%%%%%%%%%%%%%%%%%%%%%%%%%%%%%%%%%%%%%%%%%%%%%%%%%%%%%%%%%%%%%%%%%%%%%%%%%%%%%%%%%%%%%%%%%%%%%%%%%%%%%%%%%%%%%%%%%%%%%%%%%%%%%%%%%%%%%%%%%%%%%%%%%%%%%%%%%%%%%%%%%%%%%%%%%%%%%%%%%%%%%%%%%%%%%%%%%%%%%%%%%%%%%%%%%%%%%%%%%%%%%%%%%%%%%%%%%%%%%%%%%%%%%%%%%%%%%%%%%%%%%%%%%%%%%%%%%%%%%%%%%%%%%%%%%%%%%%%%%%%%%%%%%%%%%%%%%%%%%%%%%%%%%%%%%%%%%%%%%%%%%%%%%%%



\tableofcontents

\bigskip

\begin{itemize}

\item \textit{Le lien vers cet énoncé en ligne est \href{https://padlet.com/laurent_astier/oat6d9azo7rs}{ici} sur le "padlet" (Bonus 1 formation); téléchargez-le et utilisez-le pour accéder aux liens cliquables de l'énoncé.}


\medskip

\textit{Vous pouvez aussi télécharger l'énoncé à cette adresse : }


\begin{center}
\underline{\url{https://github.com/formationPythonPC-Juin/enonce-donnees/blob/master/bonus1.pdf}                                                                                               }
\end{center}







\bigskip










\item \textit{Par rapport à l'énoncé précédent, celui-ci ne propose pas forcément  de fichiers d'aides.}

\medskip

\textit{Quoiqu'il en soit, les potentiels fichiers d'aides et les fichiers corrections sont donnés en lien à la fin de chaque partie.}

\medskip
\textit{Vous réalisez donc les exercices via le niveau 1 (pas d'aide), le niveau 2 (avec aides si le fichier est fourni) ou le niveau 3 (correction donnée, il vous faut alors commenter les lignes de code pour expliquer à quoi elles servent).} 








\bigskip














\end{itemize}


\begin{comment}




\item Tous \underline{les liens soulignés} du présent énoncé sont cliquables et permettent d'accéder aux tutoriels, à des fichiers aides ou aux fichiers corrections.

\item \textit{Vous aurez auprès de vous les documents mis à disposition sur le site académique et en premier lieu : }
\begin{itemize}
 \item \href{http://pedagogie.ac-limoges.fr/physique-chimie/IMG/pdf/python-trace_de_graphe.pdf}{\underline{\texttt{Tracés de graphes avec Python}}} signalé dans ce poly par: \textsc{vademecum}
 \item \href{http://pedagogie.ac-limoges.fr/physique-chimie/IMG/pdf/python-tutoriel.pdf}{\underline{\texttt{Tutoriel pour Python}}} signalé dans ce poly par: \textsc{tutoriel}
\end{itemize}

\medskip
\item \textit{Vous traitez les exercices dans l'ordre (difficulté croissante).}

\medskip
\item \textit{Pour chaque grande partie, vous créerez un fichier (stipulé dans l'énoncé).}

\medskip
\item \textit{Chaque partie peut se traiter selon 3 niveaux de difficulté :} 

\begin{itemize}
 \item \textbf{niveau 1} : on suit les questions dans l'ordre, en s'aidant ponctuellement des tutoriels et sans copier les codes proposés.
 \item \textbf{niveau 2} : dès le début de chaque exercice, des aides à la résolution sous forme de morceaux de code sont données ; on copie-colle ces codes dans les programmes et on les complète pour répondre aux questions posées
 \item \textbf{niveau 3} : on utilise les corrections données en fin de chaque partie. Le travail alors et de \textbf{commenter} les différentes lignes du programme, c'est à dire d'expliquer à quoi sert chaque ligne.
 
 \textsc{rappels : }
 \begin{itemize}
  \item les commentaires sont des éléments du programme qui ne sont pas lus par Python, mais qui sont présents pour aider l'utilisateur du programme
  \item les commentaires sur une ligne sont introduits par \# (apparaissent en rouge sous IDLE)
  \item les commentaires sur plusieurs lignes sont introduits et terminés par un triple guillemet : \textbf{"""} (apparaissent en vert sous IDLE)
 \end{itemize}
 
\end{itemize}


\medskip
\item À chacun de choisir son niveau de difficulté ; conseil : 

\begin{itemize}
 \item Par défaut, commencez par le niveau 2, c'est ainsi que l'énoncé est rédigé. 
 
 Si vous vous sentez à l'aise, vous pouvez tenter de réaliser les exercices sans les aides (niveau 1).
 
 Si vous avez des difficultés, passez au niveau 3.
\end{itemize}


\smallskip
\item \textit{Enfin, et même si cela n'est pas précisé, pensez à compiler votre code régulièrement pour vérifier si vous répondez bien aux questions.}

\end{itemize}



\end{comment}

%%%%%%%%%%%%%%%%%%%%%%%%%%%%%%%%%%%%%%%%%%%%%%%%%%%%%%%%%%%%%%%%%%%%
%%%%%%%%%%%%%%%%%%%%%%%%%%%%%%%%%%%%%%%%%%%%%%%%%%%%%%%%%%%%%%%%%%%%
%%%%%%%%%%%%%%%%%%%%%%%%%%%%%%%%%%%%%%%%%%%%%%%%%%%%%%%%%%%%%%%%%%%%
%%%%%%%%%%%%%%%%%%%%%%%%%%%%%%%%%%%%%%%%%%%%%%%%%%%%%%%%%%%%%%%%%%%%
%%%%%%%%%%%%%%%%%%%%%%%%%%%%%%%%%%%%%%%%%%%%%%%%%%%%%%%%%%%%%%%%%%%%





\newpage

\setcounter{section}{5}


\section{Exercice 6 : Importer des données expérimentales pour les traiter par Python}


On a réalisé à l'aide d'un microcontrôleur le relevé de mesures correspondant à l'expérience suivante : 

\begin{itemize}
 \item un capteur de température permet d'acquérir la température d'un liquide qui refroidit : donnée \texttt{"température"}
 \item en parallèle, on fait calculer au microcontrôleur la résistance équivalente à un dispositif contenant une thermistance plongé dans la solution : donnée \texttt{"Réquivalente"}
\end{itemize}

\medskip

Le microcontrôleur permet d'afficher ces deux valeurs au cours du temps dans un fichier que vous avez à disposition : \href{https://github.com/formationPythonPC-Juin/enonce-donnees/blob/master/donnees}{\underline{\texttt{donnees}}}.

\medskip

On souhaite représenter le graphe \texttt{Réquivalente} en fonction de \texttt{température} dans l'espoir de pouvoir modéliser le comportement par une droite. 


\medskip


\begin{enumerate}
 \item Ouvrez le fichier  \href{https://github.com/formationPythonPC-Juin/enonce-donnees/blob/master/donnees}{\underline{\texttt{"donnees"} à cette adresse}} ; observez son contenu. 
 
 \smallskip
 Le grand nombre de lignes rend difficile le traitement par un tableur. Cliquez sur \texttt{"Raw"} puis sélectionner l'ensemble (\texttt{Ctrl+a}) et copiez-coller tout cela dans un fichier que vous pourrez appeler \texttt{donnees}.
 \item Créez un fichier \texttt{exercice6.py}. Ce fichier sera placé dans le même répertoire que le fichier de données ci-dessus. 
 
 \smallskip
 Votre fichier \texttt{exercice6.py} doit permettre :
 
 
 \begin{itemize}
  \item de placer les données dans 2 listes (qu'on pourra appeler \texttt{T} et \texttt{R})
  \item de représenter l'ensemble des points de coordonnées \texttt{température ; résistance}
  \item de modéliser l'évolution de ces points expérimentaux par une droite
 \end{itemize}

\end{enumerate}



\subsection{Récupérer les données dans des "listes"}


On souhaite créer deux listes (ou tableaux) \texttt{R} et \texttt{T} à partir des valeurs contenues dans le fichier de données.

\smallskip

\begin{enumerate}
\setcounter{enumi}{2}
\item En vous servant du \emph{vade-mecum} présent sur notre site disciplinaire \href{http://pedagogie.ac-limoges.fr/physique-chimie/IMG/pdf/python-inportation\_de\_donnees.pdf}{\underline{à ce lien}} ou encore \href{https://padlet.com/laurent_astier/oat6d9azo7rs}{\underline{ici ("Importer des données")}}, construisez les deux "listes" \texttt{T} et \texttt{R} utiles.

\smallskip
\textsc{remarque pour les utilisateurs de pandas : } Pour être précis, T et R ne sont pas des listes ni des tableaux mais un type particulier de données propre à la bibliothèque pandas. En tout cas, leurs constituants sont des nombres avec lesquels on peut faire toutes les opérations permises par Python.


\end{enumerate}




\subsection{Représentation graphique}
\begin{enumerate}
 \setcounter{enumi}{3}
 \item Représentez alors l'ensemble des points de coordonnées  \texttt{température ; résistance} sur un graphe. Donnez un titre au graphe, légendez-le ; donnez un nom aux axes. Faites afficher le graphe. Conclusion.
\end{enumerate}




\subsection{Modélisation de l'évolution des points expérimentaux}

\begin{enumerate}
 \setcounter{enumi}{4}
 \item Les points précédents semblent alignés. Trouvez et faites afficher les coefficients caractéristiques de la droite modèle. 
 \item Construisez et affichez sur le graphe la droite modèle de ce comportement. Intégrez de même une légende pour cette droite.
 \end{enumerate}








\bigskip


\begin{center}
$\blacktriangleright$ \href{https://github.com/formationPythonPC-Juin/corrections-formation/blob/master/exercice6-correction.py}{\underline{\texttt{lien vers la correction de cet exercice : exercice6-correction.py}}}$\blacktriangleleft$                                                                                                                                                                    \end{center}









%%%%%%%%%%%%%%%%%%%%%%%%%%%%%%%%%%%%%%%%%%%%%%%%%%%%%%%%%%%%%%%%%%%%
%%%%%%%%%%%%%%%%%%%%%%%%%%%%%%%%%%%%%%%%%%%%%%%%%%%%%%%%%%%%%%%%%%%%
%%%%%%%%%%%%%%%%%%%%%%%%%%%%%%%%%%%%%%%%%%%%%%%%%%%%%%%%%%%%%%%%%%%%
%%%%%%%%%%%%%%%%%%%%%%%%%%%%%%%%%%%%%%%%%%%%%%%%%%%%%%%%%%%%%%%%%%%%
%%%%%%%%%%%%%%%%%%%%%%%%%%%%%%%%%%%%%%%%%%%%%%%%%%%%%%%%%%%%%%%%%%%%















\newpage


\section{Exercices 7 et 8 : Mécanique -- Étude énergétique}




\fbox{\begin{minipage}[c]{0.8\textwidth}

\textit{Au programme de Première Spécialité : Utiliser un langage de
programmation pour effectuer le bilan énergétique d’un
système en mouvement.}
\end{minipage}}



\bigskip














\subsection{Sujet}

On souhaite simuler la trajectoire d'un solide matériel considéré comme ponctuel avant de réaliser un bilan énergétique. 

Celui-ci est lancé depuis le point (0,0) avec un angle $\alpha = 55\degres$ par rapport à l'horizontale avec une vitesse de norme $v_0 = 10\;\text{m/s}$ à l'instant $t = 0$.

La masse de l'objet est $m = 0.40\;\text{kg}$, l'accélération de la pesanteur valant $g = 9,8\;\text{N/kg}$.

Le modèle sur lequel on s'appuie intègre que les frottements de l'air sur l'objet peuvent être considérés comme une force $\vv{f} = -k\cdot \vv{v}$ et on prendra $k = 5,0\cdot 10^{-2}\;\text{SI}$.



On pose de plus $\tau = \dfrac{m}{k}$

Dans ces conditions, la vitesse est donnée par : $\vv{v} = \vv{v_0}\cdot \text{e}^{-\left(\dfrac{t}{\tau}\right)}+\tau\cdot \vv{g}\times \left(1 - \text{e}^{-\left(\dfrac{t}{\tau}\right)}\right)$ 

et le point se situe à $\vv{r} = \vv{v_0}\cdot \tau\times\left( 1 -  \text{e}^{-\left(\dfrac{t}{\tau}\right)}\right)+\tau\cdot \vv{g}\cdot t + \tau^2\cdot \vv{g}\times \left( \text{e}^{-\left(\dfrac{t}{\tau}\right)} - 1\right)$ 

soit en projection sur les axes : $\left\{ \begin{array}{ll}
                                            v_x = v_0\cdot \cos \alpha \cdot \text{e}^{-\left(t/\tau\right)}\\
                                            
                                            v_y = v_0\cdot \sin \alpha \cdot \text{e}^{-\left(t/\tau\right)} - {\tau\cdot g}\times \left(1 - \text{e}^{-\left(t/\tau\right)}\right)
                                            
                                           \end{array}\right.$






$\left\{ \begin{array}{ll}
x_M =& {v_0}\cdot \cos \alpha\cdot \tau\times\left( 1 -  \text{e}^{-\left(t/\tau\right)}\right) \\                                                                                                                                                                                           


y_M =& \left( {v_0}\cdot \sin \alpha \cdot \tau+ \tau^2\cdot {g}\right) \times\left( 1 -  \text{e}^{-\left(t/\tau\right)}\right)-\tau\cdot {g}\cdot t
\end{array}\right.$





\medskip




Pour cet exercice, vous pouvez créer un fichier, \texttt{exercice7.py}.

\medskip


\subsection{Exercice 7 : Visualisation de la trajectoire}


Ici, il va nous falloir construire les listes (ou tableaux) T, X et Y donnant les coordonnées temporelles et spatiales du point.

\medskip
On pourrait construire des listes, cela serait un peu long. On préfèrera utiliser la bibliothèque \texttt{numpy} qui permet de réaliser des calculs "naturels" sur des tableaux.

\medskip



 
\begin{enumerate}
 \item Rentrez les données de l'exercice dans des variables aux noms appropriés (les puissances de 10 se notent \texttt{e} sous Python ; de plus, la fonction \texttt{radians()} de la bibliothèque \texttt{math} convertit un angle en degrés en radians).
 
 \item L'étude se fera entre $t_0 = 0$ et $t_f = 1,7$. On choisira un pas de temps $dt = 0,1\;\text{s}$.
 


 
 
 Entrez ces valeurs dans des variables aux noms appropriés. Construire alors un \textbf{tableau} contenant tous les instants de l'étude.
 
 \textsc{aide : }
 
 Numpy (sous l'alias np) propose de construire très simplement des tableaux de valeurs ; 2 fonctions sont utiles : \texttt{np.arange()} et \texttt{np.linspace()} ; leurs paramètres sont les suivants : 
 
 \begin{itemize}
  \item \texttt{np.arange(debut, fin, intervalle\_entre\_les\_points)}
  \item \texttt{np.linspace(debut, fin, nombre\_de\_points)}
 \end{itemize}
 
 
 
 \item Construisez alors les tableaux X, Y, VX, VY tout au long du déplacement.
 
  \textsc{aide : } Numpy peut faire agir la fonction exponentielle sur tout un tableau grâce à la fonction \texttt{np.exp(...)}.
 
 
 \item Tapez le code permettant de visualiser la trajectoire du mobile sous Matplotlib.

 
 

 \end{enumerate}
 
 
 
 
 
 
 \begin{center}
 $\looparrowright$ \href{https://github.com/formationPythonPC-Juin/aides-formation/blob/master/exercice7-aide.py}{\underline{\texttt{aide à la résolution : exercice7-aide.py}}}
\end{center}
 
 








 
 
 
 
 
 
 
 
 
 
 
 
 
 
 
 
 
 
 
 
 
 \subsection{Exercice 8 : Étude énergétique}
 
 \begin{enumerate}

 
 \item \underline{Avec utilisation de Numpy} : Après avoir importé la bibliothèque Numpy, implémentez 3 listes (ou tableaux) \texttt{Ep}, \texttt{Ec} et \texttt{E} qui permettront de connaître les énergies potentielle, cinétique et mécanique du mobile au cours de son déplacement.

 
 \textsc{rappel : } $a^2$ s'écrit \texttt{a**2} en Python.
 
 
 \item Tracez alors les courbes représentant ces 3 grandeurs au cours du temps. Faites de même afficher en console les valeurs des 3 types d'énergies pour chaque instant t de la liste T.
 
 \item On constate que les valeurs d'énergies possèdent beaucoup trop de décimales par rapport aux nombres de chiffres significatifs de l'énoncé.
 
 \smallskip
 Par malheur, Python ne dispose pas de fonction toute prête pour gérer le nombre de chiffres significatifs ; il faut la fabriquer\ldots
 
 \smallskip
 Par contre, Python dispose d'une fonction qui arrondit un nombre à $n$ chiffres après la virgule, c'est la fonction \texttt{round(nbe\_à\_arrondir, n)}.
 
 \smallskip
 À titre d'exemple, construire la liste \texttt{Xarrondie} contenant les éléments de X arrondis à 1 chiffre après la virgule.
 
 \smallskip
 Faites afficher \texttt{X} et \texttt{Xarrondie}.
 
\end{enumerate}



\begin{center}
 $\looparrowright$ \href{https://github.com/formationPythonPC-Juin/aides-formation/blob/master/exercice8-aide.py}{\underline{\texttt{aide à la résolution : exercice8-aide.py}}}
\end{center}



\bigskip
\begin{center}
$\blacktriangleright$ \href{https://github.com/formationPythonPC-Juin/corrections-formation/blob/master/exercices7-8-correction.py}{\underline{\texttt{lien vers la correction de cette partie (exercices 7 et 8) : exercices7-8-correction.py}}}$\blacktriangleleft$                                                                                                                                                                    \end{center}




\bigskip

\textsc{remarque : }

En fait Numpy propose de base une fonction \texttt{round()} qui arrondit tous les éléments du tableau à un certain nombre de chiffres après la virgule. 

\medskip
Ainsi, pour répondre à la question 3, il aurait suffi de taper : 


\begin{center}
\texttt{Xarrondie = np.round(X, 1)}                                   \end{center}






%%%%%%%%%%%%%%%%%%%%%%%%%%%%%%%%%%%%%%%%%%%%%%%%%%%%%%%%%%%%%%%%%%%%%%%%%%%%%%%%%%%%%%%%%%%%%%%%%%%%%%%%%%%%%%%%%%%%%%%%%%%%%%%%%%%%%%%%%%%%%%%%%%%%%%%%%%%%%%%%%%%%%%%%%%%%%%%%%%%%%%%%%%%%%%%%%%%%%%%%%%%%%%%%%%%%%%%%%%%%%%%%%%%%%%%%%%%%%%%%%%%%%%%%%%%%%%%%%%%%%%%%%%%%%%%%%%%%%%%%%%%%%%%%%%%%%%%%%%%%%%%%%%%%%%%%%%%%%%%%%%%%%%%%%%%%%%%%%%%%%%%%%%%%%%%%%%


\newpage



\section{Exercice 9 : Onde progressive, périodicité spatiale et temporelle}



\fbox{\begin{minipage}[c]{0.8\textwidth}

\textit{Au programme de Première (Spécialité) : Simuler à
l’aide d’un langage de programmation, la propagation
d’une onde périodique.}
\end{minipage}}



\bigskip


\subsection{Sujet}


On souhaite représenter la propagation d'un signal périodique qui s'écrit sous la forme : $I(x) = I_0\times \cos (\omega\cdot t - k\cdot x + \phi )$ et visualiser la progression sur [0 ; 5]



\begin{itemize}
 \item L'étude se fait dans la région de l'espace des $x$ : [0,5]
 \item On prendra $I_0 = 3$
 \item On prendra $k = \pi$
 \item  On prendra $\phi = \dfrac{\pi}{2}$
\item On prendra $\omega = \pi$
 \end{itemize}




\subsection{Point cours}

\fbox{\begin{minipage}{0.9\textwidth}
\textsc{point cours : }

\medskip
C'est sans doute la partie la plus difficile (en terme de code) de ce que nous devrons traiter (En 2\textsuperscript{nde} et 1\textsuperscript{ère}). Le principe est le suivant : 

\begin{enumerate}
 \item On construit une courbe vide en phase d'initialisation
 \item On va remplir cette courbe par des listes x et I
 \item La liste I va varier en fonction du temps, il faudra donc qu'elle soit actualisée (mise à jour) pour chaque instant avec la fonction \texttt{set\_data()}
 \item En incluant cela dans une boucle, on peut gérer ainsi une série de graphes différents pour des instants différents
 \item \texttt{matplotlib.pyplot} dispose de la fonction \texttt{pause()} qui permet dans une boucle, d'afficher plusieurs images séparées d'un certain laps de temps, passé en paramètre de \texttt{pause()}.
\end{enumerate}

\end{minipage}}


\bigskip


\textsc{remarque : }

Matplotlib dispose d'une fonction prête à l'emploi pour réaliser des animations, c'est la fonction 

\texttt{matplotlib.animation.FuncAnimation}.

Elle ne me semble pas très intéressante pour nos élèves ; en effet, on perd le sens physique de la propagation en l'utilisant. Pour cette raison nous ne l'étudierons pas.

Il faut noter tout de même que notre méthode ne permet pas de réaliser des animations en 3D, ce que permet de faire \texttt{FuncAnimation}.










\subsection{Application}

Allez chercher  \href{https://github.com/formationPythonPC-Juin/aides-formation/blob/master/exercice9-aide.py}{\underline{le code présent ici}}, copiez-le dans un nouveau fichier que vous pourrez appeler \texttt{exercice9.py}.

\medskip
Ce code est déjà fonctionnel, vous pouvez le tester.

\medskip
Votre travail : écrire les commentaires partout où ils manquent.




\bigskip






\begin{center}
$\blacktriangleright$ \href{https://github.com/formationPythonPC-Juin/corrections-formation/blob/master/exercice9-correction.py}{\underline{\texttt{lien vers la correction de cet exercice : exercice9-correction.py}}}$\blacktriangleleft$                                                                                                                                                                    \end{center}









%%%%%%%%%%%%%%%%%%%%%%%%%%%%%%%%%%%%%%%%%%%%%%%%%%%%%%%%%%%%%%%%%%%%%%%%%%%%%%%%%%%%%%%%%
%%%%%%%%%%%%%%%%%%%%%%%%%%%%%%%%%%%%%%%%%%%%%%%%%%%%%%%%%%%%%%%%%%%%%%%%%%%%%%%%%%%%%%%%%
%%%%%%%%%%%%%%%%%%%%%%%%%%%%%%%%%%%%%%%%%%%%%%%%%%%%%%%%%%%%%%%%%%%%%%%%%%%%%%%%%%%%%%%%%
%%%%%%%%%%%%%%%%%%%%%%%%%%%%%%%%%%%%%%%%%%%%%%%%%%%%%%%%%%%%%%%%%%%%%%%%%%%%%%%%%%%%%%%%%
%%%%%%%%%%%%%%%%%%%%%%%%%%%%%%%%%%%%%%%%%%%%%%%%%%%%%%%%%%%%%%%%%%%%%%%%%%%%%%%%%%%%%%%%%

\newpage




 
 
 





















\end{document}
